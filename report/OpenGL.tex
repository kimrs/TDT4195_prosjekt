\section{OpenGL}
The program created for this part of the assignment, 
is made by modifying code given to us for the graphics part of the course. 
The color-buffers and the vertex-buffer is created in the initialization phase. 
For each frame, RenderScene is called. 
The txt-file is read once for each object everytime RenderScene is called.
The text-file is read using ReadFile.cpp.
ReadFile uses the windows library conio.h to read the file. 
The read function used to get the values, requires an integer representing a line in the .txt as input.
Parameters requiered to represent the object is extracted from that line, and placed in a array.
A pointer to the first parameter is then returned.
By reading the txt-file each time the scene is rendered, the program is able to change the scene while running.
This is done by modifying the txt-file. 
Each cube needs four parameters to be represented.
One for color, one for radius, one for the x-coordinate and one for the y-coordinate. 
After setting the values, a model-matrixfor the cube is made. 
The model-matrix consists of an identity-matrix multiplyed with a translation-matrix, multiplyed with a scale-matrix. 
The model-matrix for the cube represents the radius and position of the object. 
After multiplying the model-matrix with the mvp-matrix, a collor-buffer is bound before drawing the cube. 
The process is repeated until all objects are represented. 


\subsection{Improvements}
Today, the cubes are given color by static premade color-buffers. 
To support all colors, one buffer for each different color given in the input-file should be made. 
This solution was considered, but because of memoryleaks in the program, the values from the file where overwritten. 
Lack of experience programming in the C-language, resulted in our program not being able to support dynamically made colorbuffers.


